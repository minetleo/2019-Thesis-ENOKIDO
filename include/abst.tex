% vim: set filetype=tex :

\newcounter{keywordcnt}
\newcommand{\keyword}[1]{\addtocounter{keywordcnt}{1} \underline{\thekeywordcnt . #1} }
{\Large
\begin{center}
卒業論文2019年度(令和元年度) \\
\fbox{ \begin{tabular}{c}
\thetitle
\end{tabular} }
\end{center}
}
\section*{論文要旨}

\indent IoT通信技術の発達に伴って,構造物や工場内のモニタリングにおいて無線センサネットワークを利用した通信を用いるセンシングデータの収集が期待されている.例えば工場では,機器同士の接続は有線が多いが,これを無線に置き換えれば機器の配置換えにかかる作業時間の短縮やコスト削減が期待できる.しかし,高い信頼性が求められる制御系の工場システムにおいては,通信速度や信頼性に対する懸念から,現状,無線化は4\%程度しか進んでいない~\cite{IoT}.このような背景の下,フレーム受信後にバックオフ時間をおかずにブロードキャストを繰り返す同時送信フラッディング(CTF : Concurrent Transmission Flooding)が提案されている.
同時送信フラッディングの性能は,ノード同士の時刻同期の精度に大きく左右されるため,スケジュール機能をもったホストノードが時刻同期を担っている.しかし,ホストノードの配置によっては,ホストノードからの通信を受信できないノードが発生し,ネットワーク分断が発生する可能性がある.
そこで,本研究ではホストノードから他ノードへの到達性を考慮したホスト選択手法を提案することでネットワーク分断を減らし,同時送信フラッディングの性能向上を目指す.
具体的には,ネットワーク指標によってホスト適性が高いノードを選んだ場合の最短経路長の平均や分布,ネットワーク分断のをシミュレーションによって評価した.
その結果,従来手法よりも最短経路長平均とネットワーク分断の発生確率が改善されることが分かった.また,媒介中心性によるホスト選択が同時送信フラッディングに適していることが分かった.今後の課題としては,本研究において検討しなかったネットワーク指標におけるホスト選択が同時送信フラッディングに与える影響の評価,及び実機や実際の電波伝搬を考慮したシミュレーションによる検討等があげられる.

%あと結果と課題

\vspace{-5mm}

\section*{キーワード}\vspace{-3mm}
\keyword{同時送信フラッディング}
\keyword{ZigBee}
\keyword{CTF}
\keyword{マルチホップ}
\keyword{ネットワーク中心性}
\keyword{ダイクストラ法}



\thispagestyle{fancy}
\renewcommand{\headrulewidth}{0.0pt}
\rfoot{\vspace{-5zh}電気通信大学 情報理工学域 I\hspace{-.1em}I類 セキュリティ情報学プログラム \\ \LARGE{榎戸 菜々穂}}
