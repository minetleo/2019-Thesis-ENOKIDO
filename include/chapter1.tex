\chapter{背景と目的}

\section{研究背景}
近年,モノとモノをネットワークでつなぐことで得られるデータをモニタリングや異常検知に活用するIoT(Internet of Things)が注目されている.
IoT活用による利益は,工場分野などを中心に最大で約1300兆円にも上ると推定されている~\cite{IoT}.IoT技術の成長を支えるのは,安価で小さなセンサとそれらのデータのやり取りを可能とする無線通信技術である.
無線通信規格の一つであるZigBee~\cite{zigbee}はIoT機器によく用いられる.通信距離はおよそ30-50\si{\metre}と短いが,個々が中継局としての機能をもつため,マルチホップ通信により従来は届かなかった場所や広範囲への通信が可能である.


しかし,低遅延・高信頼な通信という点で無線通信には課題が多い.工場や製造分野ではIoT技術の活用による高い経済効果が想定されているにもかかわらず,工場内の通信方式にしめる無線通信の割合は4\%程度である~\cite{IoT}.
これは,工場内で多くの様々な機器が作動しているためである.無線通信では,複数の送信元からの電波が受信端末側で干渉した場合,受信端末がパケットの受信に失敗する.多くの機器が無線通信を行う工場では電波が干渉する可能性が高まるため,一般的に損失率は上昇する.また,機器が再送を繰り返した場合,遅延が増加する.

この問題に対し,同時送信フラッディングと呼ばれる転送方法が提案されている.
同時送信フラッディングは,一般に無線通信において性能劣化となる干渉を有効的に活用することで,受信成功率改善する手法である.
これは,干渉が起きた場合でも受信ノードにおける各送信ノードからの電波到来時間差が0.5\si{\micro\second}以内であれば復調が可能であるという発見を利用した技術である.本論文では,以後,復調が可能である干渉を建設的干渉,それ以外を破壊的干渉という.同時送信フラッディングは,ノードの時刻同期ができていれば細かい制御を必要とせずにフラッディングのみでデータをネットワーク全体に転送できる.同時送信フラッディングは2011年にFerrariらによって提唱され,2017年にソナス株式会社がUNISONetとして実用化した.具体的には橋梁での地震モニタリング,製造設備の予防保全等のための振動計測,農場での植物生育モニタリング等に活用されている~\cite{sonas}.

同時送信フラッディングがモニタリングに適している理由として,
\begin{itemize}
    \item ルーチングレスな技術でありネットワークの知識のない人であっても扱いやすいこと
    \item サンプリングタイミングを同期させているため得られたデータの解析が容易
    \item 高信頼・低遅延な通信
\end{itemize}
などがある.

同時送信フラッディングの性能は受信端末側へのパケット到達タイミングに大きく左右される.パケット到達タイミングの調整にはノード間の時刻同期が必須であり,スケジュール機能を持ったホストノードがその管理を担っている.そのため,ホストノードに障害が発生した場合,建設的干渉ではなく,破壊的干渉が生じる可能性が高い.そのため,より安定的な転送制御を実現するためには,ホストノードの耐障害性を向上させるための対策が必要である.
ホスト障害対策として先行研究~\cite{lowpower}では,全ノードがホストノードの機能をもつことで,ホストノードが故障や通信できない状態に陥っても,別のノードにフェイルオーバし、ホストノードの耐障害性を向上させていた.
ここでの新たなホストノードの選出方法は,あらかじめ全ノードが保持しているリストをラウンドロビン方式で参照するものであった.なお,リスト内にはホストノードに対応したチャネルが指定されておりホストノードの変更に伴い,転送を行うチャネルも切替わる.

\section{研究目的}
前述のように,従来手法ではリストに基づくホスト選出を行っており,決められた時間内に新ホストからのパケットを受信できなかったノードが,チャネルを切替え続けてメインのネットワークに合流できず,ネットワーク分断が生じる可能性があった.本研究では,特定のホストノードに障害が発生した場合においても,高い信頼性を維持したまま同時送信フラッディングによる転送を継続するため,到達性を考慮してホストノードを決定する手法を提案する.ここで到達性は,後述の最短経路長平均やInf数を意味する.ネットワーク上におけるノードの重要性を評価するための指標である次数中心性や媒介中心性などのネットワーク指標が高いノードをホストノードとすることで,同時送信フラッディングにおける信頼性の向上を目指す.%同時送信フラッディングにおける到達性にどのように影響するかを評価する.

ホストノードを上記のネットワーク指標で選出し,限定することで従来のラウンドロビン方式のホスト選出と比較して,最短経路長やホストノードからのスケジューリングパケットが到達しないノードが減るのではないかと考えた.その結果,ネットワーク分断や遅延が減り,同時送信フラッディングの信頼性向上が期待できる.

\section{本論文の構成}
本論文の構成は以下の通りである.
\begin{itemize}
\item{第1章 ワイヤレスセンサーネットワーク分野の背景と目的}
\item{第2章 同時送信フラッディング}
\item{第3章 到達性を考慮したホスト選択手法}
\item{第4章 性能評価}
\item{第5章 結論と今後の課題}
\end{itemize}

第2章では同時送信フラッディングやマルチホップネットワークで使われる既存技術や関連研究について述べる.第3章では同時送信フラッディングにおけるホスト選択の提案手法について述べ,第4章ではシミュレーションプログラムの詳細とそれによる提案手法の評価結果を示す.第5章では本研究で得られた結果に基づく結論と,今後の課題について述べる.

本論文における約束事は以下のとおりである.
\begin{itemize}
    \item 章,節は例のような順序で大項目から小項目へと移る. (例: 第2章,2.1)
    \item 式,図,及び表は,章単位で通し番号をつける. (例: 図 3.2)
    \item 参考文献は,文章中及び文章の末尾に文献番号で示す.
\end{itemize}
