\chapter{結論と今後の課題}

\section{結論}
本研究では,同時送信フラッディングにおけるホスト選択がランダムにラウンドロビン方式で行われていることに着目し,ホストノードをネットワーク中心性に基づいて決定する手法を提案した.同時送信フラッディングのスケジューリングを担うホストノードと,他のノードにおいて,最短経路長平均の短縮とInf数が抑制されればより高信頼・低遅延な通信ができるのではと考えたからである.ホストノードをランダムに選出した場合と,中心性に基づき選出した場合でそれぞれ1万回ずつシミュレーションを行った.
その結果,最短経路長平均とInf数の両方において従来手法よりも提案手法の方が,優れていることがわかった.
2つの提案手法に関しては,最短経路長平均では,次数中心性によるホスト選択が優れていることがわかった.しかし,ノード密度があがるにつれ,従来手法と提案手法の最短経路長平均の差異はほとんどなくなった.つまり,ノードの密度が高い状況においてはホスト選択手法に関わらず最短経路長平均が短くなることがわかった.
一方,Inf数においては媒介中心性によるホスト選択が適していることがわかった.また,Inf数においてはノード密度が高くなっても従来手法と提案手法の差が縮まらなかった.これにより,ノードの密度に関わらず,ネットワーク分断を避けるためには提案手法が有効であることがわかった.
本研究では同時送信フラッディングにおける到達性を考慮したホスト選択に関して,媒介中心性によるホスト選択が優れているという結論になった.

まとめとして従来手法と提案手法2つの関係性を図\ref{tab:5_spl}と図\ref{tab:5_inf}に示す.

\begin{table}[H]
  \caption{最短経路長平均の結果}
  \begin{tabular}{c|c|c|c} \hline\hline
      &従来手法& 媒介中心性 & 次数中心性  \\ \hline 
    50 & × & △ & 〇\\
    100& △ & 〇 & 〇 \\\hline\hline
  \end{tabular}
  \label{tab:5_spl}
\end{table}

\begin{table}[H]
  \caption{Inf数の結果}
  \begin{tabular}{c|c|c|c} \hline\hline
      &従来手法& 媒介中心性 & 次数中心性  \\ \hline 
    50 & × & 〇 & △\\
    100 & × & 〇& △ \\\hline\hline
  \end{tabular}
  \label{tab:5_inf}
\end{table}



\section{今後の課題}

    (1)実装への課題\\
    本研究においては同時送信フラッディング前のメッセージ等の交換によってノード位置やネットワークトポロジを各ノードが把握している前提とした.この作業は従来手法では行われておらず,実装に当たってはそれぞれのノードにおけるホスト候補リストの整合性の確保が課題である.
    (2)他のネットワーク指標によるホスト選出\\
    ホスト選択指標として代表的なネットワーク指標である媒介中心性と次数中心性を用いることを提案したが,これ以外にもPageRankや近接中心性などの多くのネットワーク指標がある.こうしたネットワーク指標でもホスト選択を行い,比較する必要があると考える.
    (3)適切なホスト選択手法の選択\\
    本研究において,最短経路長平均とInf数で,適しているホスト選択手法が異なった.今後はネットワークの状況に応じて適切なホスト選択手法を選択できるとよいと考える,
    (4)ノード数を増やした検討\\
    本研究では,時間の都合上,50ノードと100ノードでシミュレーションを行い,提案手法の優位性を確認した.同時送信フラッディングを用いた構造モニタリングでは47台のセンサノードで900\si{\meter}の橋梁測定の実績があるが,今後ノード数を増やした場合においても提案手法が有効か検討が必要である.
    (5)実機での検討\\
    また,本研究においては距離のみに基づく到達判定によるシミュレーションを行ったため,実機や電波伝搬を考慮したシミュレーションによる検討が必要だと思われる.




